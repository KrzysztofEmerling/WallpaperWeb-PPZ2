%% Generated by Sphinx.
\def\sphinxdocclass{report}
\documentclass[letterpaper,10pt,english]{sphinxmanual}
\ifdefined\pdfpxdimen
   \let\sphinxpxdimen\pdfpxdimen\else\newdimen\sphinxpxdimen
\fi \sphinxpxdimen=.75bp\relax
\ifdefined\pdfimageresolution
    \pdfimageresolution= \numexpr \dimexpr1in\relax/\sphinxpxdimen\relax
\fi
%% let collapsible pdf bookmarks panel have high depth per default
\PassOptionsToPackage{bookmarksdepth=5}{hyperref}

\PassOptionsToPackage{booktabs}{sphinx}
\PassOptionsToPackage{colorrows}{sphinx}

\PassOptionsToPackage{warn}{textcomp}
\usepackage[utf8]{inputenc}
\ifdefined\DeclareUnicodeCharacter
% support both utf8 and utf8x syntaxes
  \ifdefined\DeclareUnicodeCharacterAsOptional
    \def\sphinxDUC#1{\DeclareUnicodeCharacter{"#1}}
  \else
    \let\sphinxDUC\DeclareUnicodeCharacter
  \fi
  \sphinxDUC{00A0}{\nobreakspace}
  \sphinxDUC{2500}{\sphinxunichar{2500}}
  \sphinxDUC{2502}{\sphinxunichar{2502}}
  \sphinxDUC{2514}{\sphinxunichar{2514}}
  \sphinxDUC{251C}{\sphinxunichar{251C}}
  \sphinxDUC{2572}{\textbackslash}
\fi
\usepackage{cmap}
\usepackage[T1]{fontenc}
\usepackage{amsmath,amssymb,amstext}
\usepackage{babel}



\usepackage{tgtermes}
\usepackage{tgheros}
\renewcommand{\ttdefault}{txtt}



\usepackage[Bjarne]{fncychap}
\usepackage{sphinx}

\fvset{fontsize=auto}
\usepackage{geometry}


% Include hyperref last.
\usepackage{hyperref}
% Fix anchor placement for figures with captions.
\usepackage{hypcap}% it must be loaded after hyperref.
% Set up styles of URL: it should be placed after hyperref.
\urlstyle{same}

\addto\captionsenglish{\renewcommand{\contentsname}{Contents:}}

\usepackage{sphinxmessages}
\setcounter{tocdepth}{1}



\title{Dokumentacja projektu WallpaperWeb}
\date{Dec 27, 2025}
\release{1.0.0}
\author{Agnieszka Głowacka \\ Anastasiya Dorosh \\ Martyna Trębacz \\ Anna Waleczek \\ Oliwia Skucha \\ Jakub Rogoża \\ Krzysztof Emerling \\ Szymon Duda}
\newcommand{\sphinxlogo}{\vbox{}}
\renewcommand{\releasename}{Release}
\makeindex
\begin{document}

\ifdefined\shorthandoff
  \ifnum\catcode`\=\string=\active\shorthandoff{=}\fi
  \ifnum\catcode`\"=\active\shorthandoff{"}\fi
\fi

\pagestyle{empty}
\sphinxmaketitle
\pagestyle{plain}
\sphinxtableofcontents
\pagestyle{normal}
\phantomsection\label{\detokenize{index::doc}}


\sphinxAtStartPar
Add your content using \sphinxcode{\sphinxupquote{reStructuredText}} syntax. See the
\sphinxhref{https://www.sphinx-doc.org/en/master/usage/restructuredtext/index.html}{reStructuredText}
documentation for details.

\sphinxstepscope


\chapter{Python Functions}
\label{\detokenize{python/python-functions:python-functions}}\label{\detokenize{python/python-functions::doc}}\index{get\_locale() (in module app)@\spxentry{get\_locale()}\spxextra{in module app}}

\begin{fulllineitems}
\phantomsection\label{\detokenize{python/python-functions:app.get_locale}}
\pysigstartsignatures
\pysiglinewithargsret
{\sphinxcode{\sphinxupquote{app.}}\sphinxbfcode{\sphinxupquote{get\_locale}}}
{}
{}
\pysigstopsignatures
\sphinxAtStartPar
Pobiera preferowany język użytkownika zapisany w sesji.
\begin{description}
\sphinxlineitem{Returns:}
\sphinxAtStartPar
str: Kod języka (np. ‘en’ dla angielskiego), domyślnie ‘en’.

\end{description}

\end{fulllineitems}


\sphinxstepscope


\chapter{JavaScript Functions}
\label{\detokenize{js/js-functions:javascript-functions}}\label{\detokenize{js/js-functions::doc}}\index{updateStats() (built\sphinxhyphen{}in function)@\spxentry{updateStats()}\spxextra{built\sphinxhyphen{}in function}}

\begin{fulllineitems}
\phantomsection\label{\detokenize{js/js-functions:updateStats}}
\pysigstartsignatures
\pysiglinewithargsret
{\sphinxbfcode{\sphinxupquote{\DUrole{n}{updateStats}}}}
{}
{}
\pysigstopsignatures
\sphinxAtStartPar
Funkcja wyliczająca FPS i Frametime na podstawie różnicy wydajności w poprzedniej i aktualnej klatce.

\end{fulllineitems}

\index{resizeCanvas() (built\sphinxhyphen{}in function)@\spxentry{resizeCanvas()}\spxextra{built\sphinxhyphen{}in function}}

\begin{fulllineitems}
\phantomsection\label{\detokenize{js/js-functions:resizeCanvas}}
\pysigstartsignatures
\pysiglinewithargsret
{\sphinxbfcode{\sphinxupquote{\DUrole{n}{resizeCanvas}}}}
{}
{}
\pysigstopsignatures
\sphinxAtStartPar
Funkcja obsługująca automatyczne dostosowywanie rozmiaru canvasu do aktualnego rozmiaru okna przegladarki.

\end{fulllineitems}

\index{createShader() (built\sphinxhyphen{}in function)@\spxentry{createShader()}\spxextra{built\sphinxhyphen{}in function}}

\begin{fulllineitems}
\phantomsection\label{\detokenize{js/js-functions:createShader}}
\pysigstartsignatures
\pysiglinewithargsret
{\sphinxbfcode{\sphinxupquote{\DUrole{n}{createShader}}}}
{\sphinxparam{\DUrole{n}{gl}}\sphinxparamcomma \sphinxparam{\DUrole{n}{type}}\sphinxparamcomma \sphinxparam{\DUrole{n}{source}}}
{}
\pysigstopsignatures
\sphinxAtStartPar
Tworzy i kompiluje shader WebGL.
\begin{quote}\begin{description}
\sphinxlineitem{Arguments}\begin{itemize}
\item {} 
\sphinxAtStartPar
\sphinxstyleliteralstrong{\sphinxupquote{gl}} (\sphinxstylestrong{WebGL2RenderingContext}) \textendash{} Aktywny kontekst WebGL2.

\item {} 
\sphinxAtStartPar
\sphinxstyleliteralstrong{\sphinxupquote{type}} (\sphinxstylestrong{number}) \textendash{} Typ shadera (\sphinxtitleref{gl.VERTEX\_SHADER} lub \sphinxtitleref{gl.FRAGMENT\_SHADER}).

\item {} 
\sphinxAtStartPar
\sphinxstyleliteralstrong{\sphinxupquote{source}} (\sphinxstylestrong{string}) \textendash{} Kod źródłowy shadera w języku GLSL.

\end{itemize}

\sphinxlineitem{Returns}
\sphinxAtStartPar
\sphinxstylestrong{WebGLShader|null} \textendash{} \sphinxhyphen{} Skompilowany shader lub \sphinxtitleref{null} w przypadku błędu kompilacji.

\end{description}\end{quote}

\end{fulllineitems}

\index{createProgram() (built\sphinxhyphen{}in function)@\spxentry{createProgram()}\spxextra{built\sphinxhyphen{}in function}}

\begin{fulllineitems}
\phantomsection\label{\detokenize{js/js-functions:createProgram}}
\pysigstartsignatures
\pysiglinewithargsret
{\sphinxbfcode{\sphinxupquote{\DUrole{n}{createProgram}}}}
{\sphinxparam{\DUrole{n}{gl}}\sphinxparamcomma \sphinxparam{\DUrole{n}{vertexShader}}\sphinxparamcomma \sphinxparam{\DUrole{n}{fragmentShader}}}
{}
\pysigstopsignatures
\sphinxAtStartPar
Tworzy i linkuje program WebGL z shaderów wierzchołków i fragmentów.
\begin{quote}\begin{description}
\sphinxlineitem{Arguments}\begin{itemize}
\item {} 
\sphinxAtStartPar
\sphinxstyleliteralstrong{\sphinxupquote{gl}} (\sphinxstylestrong{WebGL2RenderingContext}) \textendash{} Aktywny kontekst WebGL2.

\item {} 
\sphinxAtStartPar
\sphinxstyleliteralstrong{\sphinxupquote{vertexShader}} (\sphinxstylestrong{WebGLShader}) \textendash{} Skompilowany shader wierzchołków.

\item {} 
\sphinxAtStartPar
\sphinxstyleliteralstrong{\sphinxupquote{fragmentShader}} (\sphinxstylestrong{WebGLShader}) \textendash{} Skompilowany shader fragmentów.

\end{itemize}

\sphinxlineitem{Returns}
\sphinxAtStartPar
\sphinxstylestrong{WebGLProgram|null} \textendash{} \sphinxhyphen{} Zlinkowany program WebGL lub \sphinxtitleref{null} w przypadku błędu linkowania.

\end{description}\end{quote}

\end{fulllineitems}

\index{loadShaderSource() (built\sphinxhyphen{}in function)@\spxentry{loadShaderSource()}\spxextra{built\sphinxhyphen{}in function}}

\begin{fulllineitems}
\phantomsection\label{\detokenize{js/js-functions:loadShaderSource}}
\pysigstartsignatures
\pysiglinewithargsret
{\sphinxbfcode{\sphinxupquote{\DUrole{n}{loadShaderSource}}}}
{\sphinxparam{\DUrole{n}{name}}}
{}
\pysigstopsignatures
\sphinxAtStartPar
Asynchronicznie ładuje kod źródłowy shadera z pliku.
\begin{quote}\begin{description}
\sphinxlineitem{Arguments}\begin{itemize}
\item {} 
\sphinxAtStartPar
\sphinxstyleliteralstrong{\sphinxupquote{name}} (\sphinxstylestrong{string}) \textendash{} Nazwa pliku shadera znajdującego się w katalogu \sphinxtitleref{/static/shaders/}.

\end{itemize}

\sphinxlineitem{Throws}
\sphinxAtStartPar
\sphinxstyleliteralstrong{\sphinxupquote{Error}} \textendash{} \begin{itemize}
\item {} 
\sphinxAtStartPar
Gdy plik shadera nie może zostać załadowany.

\end{itemize}


\sphinxlineitem{Returns}
\sphinxAtStartPar
\sphinxstylestrong{Promise.\textless{}string\textgreater{}} \textendash{} \sphinxhyphen{} Kod źródłowy shadera w formacie tekstowym.

\end{description}\end{quote}

\end{fulllineitems}

\index{init() (built\sphinxhyphen{}in function)@\spxentry{init()}\spxextra{built\sphinxhyphen{}in function}}

\begin{fulllineitems}
\phantomsection\label{\detokenize{js/js-functions:init}}
\pysigstartsignatures
\pysiglinewithargsret
{\sphinxbfcode{\sphinxupquote{\DUrole{n}{init}}}}
{}
{}
\pysigstopsignatures
\sphinxAtStartPar
Asynchronicznie ładuje wszystkie shadery wykorzystywane w aplikacji.
\begin{quote}\begin{description}
\sphinxlineitem{Returns}
\sphinxAtStartPar
\sphinxstylestrong{Promise.\textless{}\{vertexShaderSource: string, fragmentShaderSource: string, fragmentAsciiShaderSource: string, fragmentFXAASource: string\}\textgreater{}} \textendash{} \sphinxhyphen{} Obiekt zawierający kody źródłowe wszystkich shaderów.

\end{description}\end{quote}

\end{fulllineitems}

\index{createTextureFromImage() (built\sphinxhyphen{}in function)@\spxentry{createTextureFromImage()}\spxextra{built\sphinxhyphen{}in function}}

\begin{fulllineitems}
\phantomsection\label{\detokenize{js/js-functions:createTextureFromImage}}
\pysigstartsignatures
\pysiglinewithargsret
{\sphinxbfcode{\sphinxupquote{\DUrole{n}{createTextureFromImage}}}}
{\sphinxparam{\DUrole{n}{gl}}\sphinxparamcomma \sphinxparam{\DUrole{n}{program}}\sphinxparamcomma \sphinxparam{\DUrole{n}{image}}\sphinxparamcomma \sphinxparam{\DUrole{n}{textureSlot}}\sphinxparamcomma \sphinxparam{\DUrole{n}{uniformName}}}
{}
\pysigstopsignatures
\sphinxAtStartPar
Funkcja tworząca teksturę 2D z wczytanego obrazu.
\begin{quote}\begin{description}
\sphinxlineitem{Arguments}\begin{itemize}
\item {} 
\sphinxAtStartPar
\sphinxstyleliteralstrong{\sphinxupquote{gl}} (\sphinxstylestrong{WebGL2RenderingContext}) \textendash{} wskaźnik na kontekst gl.

\item {} 
\sphinxAtStartPar
\sphinxstyleliteralstrong{\sphinxupquote{program}} (\sphinxstylestrong{WebGLProgram}) \textendash{} wskaźnik na program.

\item {} 
\sphinxAtStartPar
\sphinxstyleliteralstrong{\sphinxupquote{image}} (\sphinxstylestrong{TexImageSource}) \textendash{} plik obrazu.

\item {} 
\sphinxAtStartPar
\sphinxstyleliteralstrong{\sphinxupquote{textureSlot}} (\sphinxstylestrong{number}) \textendash{} wskaźnik na slot w który ładujemy teksturę.

\item {} 
\sphinxAtStartPar
\sphinxstyleliteralstrong{\sphinxupquote{uniformName}} (\sphinxstylestrong{string}) \textendash{} nazwa uniformu pod który podpinamy teksturę.

\end{itemize}

\sphinxlineitem{Returns}
\sphinxAtStartPar
\sphinxstylestrong{WebGLTexture} \textendash{} \sphinxhyphen{} obiekt tekstury

\end{description}\end{quote}

\end{fulllineitems}

\index{createRenderTarget() (built\sphinxhyphen{}in function)@\spxentry{createRenderTarget()}\spxextra{built\sphinxhyphen{}in function}}

\begin{fulllineitems}
\phantomsection\label{\detokenize{js/js-functions:createRenderTarget}}
\pysigstartsignatures
\pysiglinewithargsret
{\sphinxbfcode{\sphinxupquote{\DUrole{n}{createRenderTarget}}}}
{\sphinxparam{\DUrole{n}{gl}}\sphinxparamcomma \sphinxparam{\DUrole{n}{width}}\sphinxparamcomma \sphinxparam{\DUrole{n}{height}}}
{}
\pysigstopsignatures
\sphinxAtStartPar
Tworzy bufor ramki (Framebuffer) z teksturą jako celem renderowania.
\begin{quote}\begin{description}
\sphinxlineitem{Arguments}\begin{itemize}
\item {} 
\sphinxAtStartPar
\sphinxstyleliteralstrong{\sphinxupquote{gl}} (\sphinxstylestrong{WebGL2RenderingContext}) \textendash{} Aktywny kontekst WebGL2.

\item {} 
\sphinxAtStartPar
\sphinxstyleliteralstrong{\sphinxupquote{width}} (\sphinxstylestrong{number}) \textendash{} Szerokość tekstury render targetu w pikselach.

\item {} 
\sphinxAtStartPar
\sphinxstyleliteralstrong{\sphinxupquote{height}} (\sphinxstylestrong{number}) \textendash{} Wysokość tekstury render targetu w pikselach.

\end{itemize}

\sphinxlineitem{Returns}
\sphinxAtStartPar
\sphinxstylestrong{Object} \textendash{} \sphinxhyphen{} Obiekt zawierający framebuffer oraz powiązaną z nim teksturę.

\end{description}\end{quote}

\end{fulllineitems}

\index{toggleScene() (built\sphinxhyphen{}in function)@\spxentry{toggleScene()}\spxextra{built\sphinxhyphen{}in function}}

\begin{fulllineitems}
\phantomsection\label{\detokenize{js/js-functions:toggleScene}}
\pysigstartsignatures
\pysiglinewithargsret
{\sphinxbfcode{\sphinxupquote{\DUrole{n}{toggleScene}}}}
{}
{}
\pysigstopsignatures
\sphinxAtStartPar
Funkcja obsługująca zmiane sceny.

\end{fulllineitems}

\index{vector1i() (built\sphinxhyphen{}in function)@\spxentry{vector1i()}\spxextra{built\sphinxhyphen{}in function}}

\begin{fulllineitems}
\phantomsection\label{\detokenize{js/js-functions:vector1i}}
\pysigstartsignatures
\pysiglinewithargsret
{\sphinxbfcode{\sphinxupquote{\DUrole{n}{vector1i}}}}
{\sphinxparam{\DUrole{n}{x\_}}}
{}
\pysigstopsignatures
\sphinxAtStartPar
Funkcja tworząca jednowymiarowy wektor INT.
\begin{quote}\begin{description}
\sphinxlineitem{Arguments}\begin{itemize}
\item {} 
\sphinxAtStartPar
\sphinxstyleliteralstrong{\sphinxupquote{x\_}} (\sphinxstylestrong{int}) \textendash{} wartość x

\end{itemize}

\sphinxlineitem{Returns}
\sphinxAtStartPar
wektor jednowymiarowy w formie listy {[}x{]}

\end{description}\end{quote}

\end{fulllineitems}

\index{vector2i() (built\sphinxhyphen{}in function)@\spxentry{vector2i()}\spxextra{built\sphinxhyphen{}in function}}

\begin{fulllineitems}
\phantomsection\label{\detokenize{js/js-functions:vector2i}}
\pysigstartsignatures
\pysiglinewithargsret
{\sphinxbfcode{\sphinxupquote{\DUrole{n}{vector2i}}}}
{\sphinxparam{\DUrole{n}{x\_}}\sphinxparamcomma \sphinxparam{\DUrole{n}{y\_}}}
{}
\pysigstopsignatures
\sphinxAtStartPar
Funkcja tworząca dwuwymiarowy wektor INT.
\begin{quote}\begin{description}
\sphinxlineitem{Arguments}\begin{itemize}
\item {} 
\sphinxAtStartPar
\sphinxstyleliteralstrong{\sphinxupquote{x\_}} (\sphinxstylestrong{int}) \textendash{} wartość x

\item {} 
\sphinxAtStartPar
\sphinxstyleliteralstrong{\sphinxupquote{y\_}} (\sphinxstylestrong{int}) \textendash{} wartość y

\end{itemize}

\sphinxlineitem{Returns}
\sphinxAtStartPar
wektor dwuwymiarowy w formie listy {[}x, y{]}

\end{description}\end{quote}

\end{fulllineitems}

\index{vector3i() (built\sphinxhyphen{}in function)@\spxentry{vector3i()}\spxextra{built\sphinxhyphen{}in function}}

\begin{fulllineitems}
\phantomsection\label{\detokenize{js/js-functions:vector3i}}
\pysigstartsignatures
\pysiglinewithargsret
{\sphinxbfcode{\sphinxupquote{\DUrole{n}{vector3i}}}}
{\sphinxparam{\DUrole{n}{x\_}}\sphinxparamcomma \sphinxparam{\DUrole{n}{y\_}}\sphinxparamcomma \sphinxparam{\DUrole{n}{z\_}}}
{}
\pysigstopsignatures
\sphinxAtStartPar
Funkcja tworząca trójwymiarowy wektor INT.
\begin{quote}\begin{description}
\sphinxlineitem{Arguments}\begin{itemize}
\item {} 
\sphinxAtStartPar
\sphinxstyleliteralstrong{\sphinxupquote{x\_}} (\sphinxstylestrong{int}) \textendash{} wartość x

\item {} 
\sphinxAtStartPar
\sphinxstyleliteralstrong{\sphinxupquote{y\_}} (\sphinxstylestrong{int}) \textendash{} wartość y

\item {} 
\sphinxAtStartPar
\sphinxstyleliteralstrong{\sphinxupquote{z\_}} (\sphinxstylestrong{int}) \textendash{} wartość z

\end{itemize}

\sphinxlineitem{Returns}
\sphinxAtStartPar
wektor trójwymiarowy w formie listy {[}x, y, z{]}

\end{description}\end{quote}

\end{fulllineitems}

\index{vector4i() (built\sphinxhyphen{}in function)@\spxentry{vector4i()}\spxextra{built\sphinxhyphen{}in function}}

\begin{fulllineitems}
\phantomsection\label{\detokenize{js/js-functions:vector4i}}
\pysigstartsignatures
\pysiglinewithargsret
{\sphinxbfcode{\sphinxupquote{\DUrole{n}{vector4i}}}}
{\sphinxparam{\DUrole{n}{x\_}}\sphinxparamcomma \sphinxparam{\DUrole{n}{y\_}}\sphinxparamcomma \sphinxparam{\DUrole{n}{z\_}}\sphinxparamcomma \sphinxparam{\DUrole{n}{t\_}}}
{}
\pysigstopsignatures
\sphinxAtStartPar
Funkcja tworząca czterowymiarowy wektor INT.
\begin{quote}\begin{description}
\sphinxlineitem{Arguments}\begin{itemize}
\item {} 
\sphinxAtStartPar
\sphinxstyleliteralstrong{\sphinxupquote{x\_}} (\sphinxstylestrong{int}) \textendash{} wartość x

\item {} 
\sphinxAtStartPar
\sphinxstyleliteralstrong{\sphinxupquote{y\_}} (\sphinxstylestrong{int}) \textendash{} wartość y

\item {} 
\sphinxAtStartPar
\sphinxstyleliteralstrong{\sphinxupquote{z\_}} (\sphinxstylestrong{int}) \textendash{} wartość z

\item {} 
\sphinxAtStartPar
\sphinxstyleliteralstrong{\sphinxupquote{t\_}} (\sphinxstylestrong{int}) \textendash{} wartość t

\end{itemize}

\sphinxlineitem{Returns}
\sphinxAtStartPar
wektor czterowymiarowy w formie listy {[}x, y, z, t{]}

\end{description}\end{quote}

\end{fulllineitems}

\index{vector1f() (built\sphinxhyphen{}in function)@\spxentry{vector1f()}\spxextra{built\sphinxhyphen{}in function}}

\begin{fulllineitems}
\phantomsection\label{\detokenize{js/js-functions:vector1f}}
\pysigstartsignatures
\pysiglinewithargsret
{\sphinxbfcode{\sphinxupquote{\DUrole{n}{vector1f}}}}
{\sphinxparam{\DUrole{n}{x\_}}}
{}
\pysigstopsignatures
\sphinxAtStartPar
Funkcja tworząca jednowymiarowy wektor FLOAT.
\begin{quote}\begin{description}
\sphinxlineitem{Arguments}\begin{itemize}
\item {} 
\sphinxAtStartPar
\sphinxstyleliteralstrong{\sphinxupquote{x\_}} (\sphinxstylestrong{float}) \textendash{} wartość x

\end{itemize}

\sphinxlineitem{Returns}
\sphinxAtStartPar
wektor jednowymiarowy w formie listy {[}x{]}

\end{description}\end{quote}

\end{fulllineitems}

\index{vector2f() (built\sphinxhyphen{}in function)@\spxentry{vector2f()}\spxextra{built\sphinxhyphen{}in function}}

\begin{fulllineitems}
\phantomsection\label{\detokenize{js/js-functions:vector2f}}
\pysigstartsignatures
\pysiglinewithargsret
{\sphinxbfcode{\sphinxupquote{\DUrole{n}{vector2f}}}}
{\sphinxparam{\DUrole{n}{x\_}}\sphinxparamcomma \sphinxparam{\DUrole{n}{y\_}}}
{}
\pysigstopsignatures
\sphinxAtStartPar
Funkcja tworząca dwuwymiarowy wektor FLOAT.
\begin{quote}\begin{description}
\sphinxlineitem{Arguments}\begin{itemize}
\item {} 
\sphinxAtStartPar
\sphinxstyleliteralstrong{\sphinxupquote{x\_}} (\sphinxstylestrong{float}) \textendash{} wartość x

\item {} 
\sphinxAtStartPar
\sphinxstyleliteralstrong{\sphinxupquote{y\_}} (\sphinxstylestrong{float}) \textendash{} wartość y

\end{itemize}

\sphinxlineitem{Returns}
\sphinxAtStartPar
wektor dwuwymiarowy w formie listy {[}x, y{]}

\end{description}\end{quote}

\end{fulllineitems}

\index{vector3f() (built\sphinxhyphen{}in function)@\spxentry{vector3f()}\spxextra{built\sphinxhyphen{}in function}}

\begin{fulllineitems}
\phantomsection\label{\detokenize{js/js-functions:vector3f}}
\pysigstartsignatures
\pysiglinewithargsret
{\sphinxbfcode{\sphinxupquote{\DUrole{n}{vector3f}}}}
{\sphinxparam{\DUrole{n}{x\_}}\sphinxparamcomma \sphinxparam{\DUrole{n}{y\_}}\sphinxparamcomma \sphinxparam{\DUrole{n}{z\_}}}
{}
\pysigstopsignatures
\sphinxAtStartPar
Funkcja tworząca trójwymiarowy wektor FLOAT.
\begin{quote}\begin{description}
\sphinxlineitem{Arguments}\begin{itemize}
\item {} 
\sphinxAtStartPar
\sphinxstyleliteralstrong{\sphinxupquote{x\_}} (\sphinxstylestrong{float}) \textendash{} wartość x

\item {} 
\sphinxAtStartPar
\sphinxstyleliteralstrong{\sphinxupquote{y\_}} (\sphinxstylestrong{float}) \textendash{} wartość y

\item {} 
\sphinxAtStartPar
\sphinxstyleliteralstrong{\sphinxupquote{z\_}} (\sphinxstylestrong{float}) \textendash{} wartość z

\end{itemize}

\sphinxlineitem{Returns}
\sphinxAtStartPar
wektor trójwymiarowy w formie listy {[}x, y, z{]}

\end{description}\end{quote}

\end{fulllineitems}

\index{vector4f() (built\sphinxhyphen{}in function)@\spxentry{vector4f()}\spxextra{built\sphinxhyphen{}in function}}

\begin{fulllineitems}
\phantomsection\label{\detokenize{js/js-functions:vector4f}}
\pysigstartsignatures
\pysiglinewithargsret
{\sphinxbfcode{\sphinxupquote{\DUrole{n}{vector4f}}}}
{\sphinxparam{\DUrole{n}{x\_}}\sphinxparamcomma \sphinxparam{\DUrole{n}{y\_}}\sphinxparamcomma \sphinxparam{\DUrole{n}{z\_}}\sphinxparamcomma \sphinxparam{\DUrole{n}{t\_}}}
{}
\pysigstopsignatures
\sphinxAtStartPar
Funkcja tworząca czterowymiarowy wektor FLOAT.
\begin{quote}\begin{description}
\sphinxlineitem{Arguments}\begin{itemize}
\item {} 
\sphinxAtStartPar
\sphinxstyleliteralstrong{\sphinxupquote{x\_}} (\sphinxstylestrong{float}) \textendash{} wartość x

\item {} 
\sphinxAtStartPar
\sphinxstyleliteralstrong{\sphinxupquote{y\_}} (\sphinxstylestrong{float}) \textendash{} wartość y

\item {} 
\sphinxAtStartPar
\sphinxstyleliteralstrong{\sphinxupquote{z\_}} (\sphinxstylestrong{float}) \textendash{} wartość z

\item {} 
\sphinxAtStartPar
\sphinxstyleliteralstrong{\sphinxupquote{t\_}} (\sphinxstylestrong{float}) \textendash{} wartość t

\end{itemize}

\sphinxlineitem{Returns}
\sphinxAtStartPar
wektor czterowymiarowy w formie listy {[}x, y, z, t{]}

\end{description}\end{quote}

\end{fulllineitems}

\index{normalize() (built\sphinxhyphen{}in function)@\spxentry{normalize()}\spxextra{built\sphinxhyphen{}in function}}

\begin{fulllineitems}
\phantomsection\label{\detokenize{js/js-functions:normalize}}
\pysigstartsignatures
\pysiglinewithargsret
{\sphinxbfcode{\sphinxupquote{\DUrole{n}{normalize}}}}
{\sphinxparam{\DUrole{n}{array}}}
{}
\pysigstopsignatures
\sphinxAtStartPar
Funkcja normalizująca każdą wartość w liście do zakresu {[}0, 1{]}.
\begin{quote}\begin{description}
\sphinxlineitem{Arguments}\begin{itemize}
\item {} 
\sphinxAtStartPar
\sphinxstyleliteralstrong{\sphinxupquote{array}} (\sphinxstylestrong{list}) \textendash{} tablica

\end{itemize}

\sphinxlineitem{Returns}
\sphinxAtStartPar
znormalizowana tablica.

\end{description}\end{quote}

\end{fulllineitems}

\index{starsGenerator() (built\sphinxhyphen{}in function)@\spxentry{starsGenerator()}\spxextra{built\sphinxhyphen{}in function}}

\begin{fulllineitems}
\phantomsection\label{\detokenize{js/js-functions:starsGenerator}}
\pysigstartsignatures
\pysiglinewithargsret
{\sphinxbfcode{\sphinxupquote{\DUrole{n}{starsGenerator}}}}
{\sphinxparam{\DUrole{n}{seed}}\sphinxparamcomma \sphinxparam{\DUrole{n}{minDistance}}\sphinxparamcomma \sphinxparam{\DUrole{n}{K}}}
{}
\pysigstopsignatures
\sphinxAtStartPar
Funkcja obsługująca pobieranie wartości do generatora gwiazd.
\begin{quote}\begin{description}
\sphinxlineitem{Arguments}\begin{itemize}
\item {} 
\sphinxAtStartPar
\sphinxstyleliteralstrong{\sphinxupquote{seed}} (\sphinxstylestrong{int}) \textendash{} ziarno

\item {} 
\sphinxAtStartPar
\sphinxstyleliteralstrong{\sphinxupquote{minDistance}} (\sphinxstylestrong{int}) \textendash{} minimalna odleglość pomiędzy dwoma punktami

\item {} 
\sphinxAtStartPar
\sphinxstyleliteralstrong{\sphinxupquote{K}} (\sphinxstylestrong{int}) \textendash{} ilość prób podjęta do znalezienia pasującego punktu

\end{itemize}

\sphinxlineitem{Returns}
\sphinxAtStartPar
wypłaszczony wektor zawierający współrzędne gwiazd {[}x1, y1, x2, y2, …, xn, yn{]}.

\end{description}\end{quote}

\end{fulllineitems}

\index{gaussian() (built\sphinxhyphen{}in function)@\spxentry{gaussian()}\spxextra{built\sphinxhyphen{}in function}}

\begin{fulllineitems}
\phantomsection\label{\detokenize{js/js-functions:gaussian}}
\pysigstartsignatures
\pysiglinewithargsret
{\sphinxbfcode{\sphinxupquote{\DUrole{n}{gaussian}}}}
{\sphinxparam{\DUrole{n}{x}}\sphinxparamcomma \sphinxparam{\DUrole{n}{sigma}}}
{}
\pysigstopsignatures
\sphinxAtStartPar
Funkcja pomocnicza do wyznaczania wartości jednowymiarowej funkcji Gaussa.
\begin{quote}\begin{description}
\sphinxlineitem{Arguments}\begin{itemize}
\item {} 
\sphinxAtStartPar
\sphinxstyleliteralstrong{\sphinxupquote{x}} (\sphinxstylestrong{int}) \textendash{} wartość

\item {} 
\sphinxAtStartPar
\sphinxstyleliteralstrong{\sphinxupquote{sigma}} (\sphinxstylestrong{float}) \textendash{} rozmycie sigma

\end{itemize}

\sphinxlineitem{Returns}
\sphinxAtStartPar
wartość funkcji Gaussa.

\end{description}\end{quote}

\end{fulllineitems}

\index{gaussianBlur() (built\sphinxhyphen{}in function)@\spxentry{gaussianBlur()}\spxextra{built\sphinxhyphen{}in function}}

\begin{fulllineitems}
\phantomsection\label{\detokenize{js/js-functions:gaussianBlur}}
\pysigstartsignatures
\pysiglinewithargsret
{\sphinxbfcode{\sphinxupquote{\DUrole{n}{gaussianBlur}}}}
{\sphinxparam{\DUrole{n}{kernelSize\_handler}}\sphinxparamcomma \sphinxparam{\DUrole{n}{intensity\_handler}}}
{}
\pysigstopsignatures
\sphinxAtStartPar
Funkcja obsługująca pobieranie wartości do shadera Gaussian Blur.
\begin{quote}\begin{description}
\sphinxlineitem{Arguments}\begin{itemize}
\item {} 
\sphinxAtStartPar
\sphinxstyleliteralstrong{\sphinxupquote{kernelSize\_handler}} (\sphinxstylestrong{int}) \textendash{} wielkość kernela

\item {} 
\sphinxAtStartPar
\sphinxstyleliteralstrong{\sphinxupquote{intensity\_handler}} (\sphinxstylestrong{float}) \textendash{} intensywność efektu

\end{itemize}

\sphinxlineitem{Returns}
\sphinxAtStartPar
gotowe wartości dla shadera.

\end{description}\end{quote}

\end{fulllineitems}

\index{bloom() (built\sphinxhyphen{}in function)@\spxentry{bloom()}\spxextra{built\sphinxhyphen{}in function}}

\begin{fulllineitems}
\phantomsection\label{\detokenize{js/js-functions:bloom}}
\pysigstartsignatures
\pysiglinewithargsret
{\sphinxbfcode{\sphinxupquote{\DUrole{n}{bloom}}}}
{\sphinxparam{\DUrole{n}{bloomIntensity\_handler}}\sphinxparamcomma \sphinxparam{\DUrole{n}{bloomKernelSize\_handler}}}
{}
\pysigstopsignatures
\sphinxAtStartPar
Funkcja obsługująca pobieranie wartości do shadera Bloom.
\begin{quote}\begin{description}
\sphinxlineitem{Arguments}\begin{itemize}
\item {} 
\sphinxAtStartPar
\sphinxstyleliteralstrong{\sphinxupquote{bloomIntensity\_handler}} (\sphinxstylestrong{float}) \textendash{} intensywność efektu

\item {} 
\sphinxAtStartPar
\sphinxstyleliteralstrong{\sphinxupquote{bloomKernelSize\_handler}} (\sphinxstylestrong{int}) \textendash{} wielkość kernela

\end{itemize}

\sphinxlineitem{Returns}
\sphinxAtStartPar
gotowe wartości dla shadera.

\end{description}\end{quote}

\end{fulllineitems}

\index{isChecked() (built\sphinxhyphen{}in function)@\spxentry{isChecked()}\spxextra{built\sphinxhyphen{}in function}}

\begin{fulllineitems}
\phantomsection\label{\detokenize{js/js-functions:isChecked}}
\pysigstartsignatures
\pysiglinewithargsret
{\sphinxbfcode{\sphinxupquote{\DUrole{n}{isChecked}}}}
{\sphinxparam{\DUrole{n}{element}}}
{}
\pysigstopsignatures
\sphinxAtStartPar
Sprawdza czy przycisk jest wciśnięty.
\begin{quote}\begin{description}
\sphinxlineitem{Arguments}\begin{itemize}
\item {} 
\sphinxAtStartPar
\sphinxstyleliteralstrong{\sphinxupquote{element}} (\sphinxstylestrong{object}) \textendash{} obiekt DOM

\end{itemize}

\sphinxlineitem{Returns}
\sphinxAtStartPar
boolean true/false.

\end{description}\end{quote}

\end{fulllineitems}

\index{sliderValue() (built\sphinxhyphen{}in function)@\spxentry{sliderValue()}\spxextra{built\sphinxhyphen{}in function}}

\begin{fulllineitems}
\phantomsection\label{\detokenize{js/js-functions:sliderValue}}
\pysigstartsignatures
\pysiglinewithargsret
{\sphinxbfcode{\sphinxupquote{\DUrole{n}{sliderValue}}}}
{\sphinxparam{\DUrole{n}{slider}}\sphinxparamcomma \sphinxparam{\DUrole{n}{input}}}
{}
\pysigstopsignatures
\sphinxAtStartPar
Funkcja pilnująca by wprowadzana wartość nie wchodziła poza zakres \textless{}min, max\textgreater{}.
\begin{quote}\begin{description}
\sphinxlineitem{Arguments}\begin{itemize}
\item {} 
\sphinxAtStartPar
\sphinxstyleliteralstrong{\sphinxupquote{slider}} (\sphinxstylestrong{object}) \textendash{} \begin{itemize}
\item {} 
\sphinxAtStartPar
input type Range

\end{itemize}


\item {} 
\sphinxAtStartPar
\sphinxstyleliteralstrong{\sphinxupquote{input}} (\sphinxstylestrong{object}) \textendash{} input type Number

\end{itemize}

\end{description}\end{quote}

\end{fulllineitems}

\index{restoreDefault() (built\sphinxhyphen{}in function)@\spxentry{restoreDefault()}\spxextra{built\sphinxhyphen{}in function}}

\begin{fulllineitems}
\phantomsection\label{\detokenize{js/js-functions:restoreDefault}}
\pysigstartsignatures
\pysiglinewithargsret
{\sphinxbfcode{\sphinxupquote{\DUrole{n}{restoreDefault}}}}
{\sphinxparam{\DUrole{n}{input}}}
{}
\pysigstopsignatures
\sphinxAtStartPar
Funkcja ustawiająca wartość na minimum jeżeli zostanie całkowicie usunięta z text area.
\begin{quote}\begin{description}
\sphinxlineitem{Arguments}\begin{itemize}
\item {} 
\sphinxAtStartPar
\sphinxstyleliteralstrong{\sphinxupquote{input}} (\sphinxstylestrong{object}) \textendash{} input type Number

\end{itemize}

\end{description}\end{quote}

\end{fulllineitems}

\index{inputValue() (built\sphinxhyphen{}in function)@\spxentry{inputValue()}\spxextra{built\sphinxhyphen{}in function}}

\begin{fulllineitems}
\phantomsection\label{\detokenize{js/js-functions:inputValue}}
\pysigstartsignatures
\pysiglinewithargsret
{\sphinxbfcode{\sphinxupquote{\DUrole{n}{inputValue}}}}
{\sphinxparam{\DUrole{n}{slider}}\sphinxparamcomma \sphinxparam{\DUrole{n}{input}}}
{}
\pysigstopsignatures
\sphinxAtStartPar
Funkcja ustawiający input.value takie samo jak w sliderze.
\begin{quote}\begin{description}
\sphinxlineitem{Arguments}\begin{itemize}
\item {} 
\sphinxAtStartPar
\sphinxstyleliteralstrong{\sphinxupquote{slider}} (\sphinxstylestrong{object}) \textendash{} input type Range

\item {} 
\sphinxAtStartPar
\sphinxstyleliteralstrong{\sphinxupquote{input}} (\sphinxstylestrong{object}) \textendash{} input type Number

\end{itemize}

\end{description}\end{quote}

\end{fulllineitems}

\index{inputValidation() (built\sphinxhyphen{}in function)@\spxentry{inputValidation()}\spxextra{built\sphinxhyphen{}in function}}

\begin{fulllineitems}
\phantomsection\label{\detokenize{js/js-functions:inputValidation}}
\pysigstartsignatures
\pysiglinewithargsret
{\sphinxbfcode{\sphinxupquote{\DUrole{n}{inputValidation}}}}
{\sphinxparam{\DUrole{n}{input}}}
{}
\pysigstopsignatures
\sphinxAtStartPar
Funkcja pilnująca żeby w inpucie nie można było przekroczyć wartości minimalnej i maksymalnej.
\begin{quote}\begin{description}
\sphinxlineitem{Arguments}\begin{itemize}
\item {} 
\sphinxAtStartPar
\sphinxstyleliteralstrong{\sphinxupquote{input}} (\sphinxstylestrong{object}) \textendash{} input type Number

\end{itemize}

\end{description}\end{quote}

\end{fulllineitems}

\index{saveSessionData() (built\sphinxhyphen{}in function)@\spxentry{saveSessionData()}\spxextra{built\sphinxhyphen{}in function}}

\begin{fulllineitems}
\phantomsection\label{\detokenize{js/js-functions:saveSessionData}}
\pysigstartsignatures
\pysiglinewithargsret
{\sphinxbfcode{\sphinxupquote{\DUrole{n}{saveSessionData}}}}
{}
{}
\pysigstopsignatures
\sphinxAtStartPar
Funkcja zapisująca dane sesji.

\end{fulllineitems}

\index{fetchSceneValues() (built\sphinxhyphen{}in function)@\spxentry{fetchSceneValues()}\spxextra{built\sphinxhyphen{}in function}}

\begin{fulllineitems}
\phantomsection\label{\detokenize{js/js-functions:fetchSceneValues}}
\pysigstartsignatures
\pysiglinewithargsret
{\sphinxbfcode{\sphinxupquote{\DUrole{n}{fetchSceneValues}}}}
{}
{}
\pysigstopsignatures
\sphinxAtStartPar
Funkcja pobierająca aktualne wartości w danej scenie.
\begin{quote}\begin{description}
\sphinxlineitem{Returns}
\sphinxAtStartPar
array

\end{description}\end{quote}

\end{fulllineitems}

\index{updateSceneValues() (built\sphinxhyphen{}in function)@\spxentry{updateSceneValues()}\spxextra{built\sphinxhyphen{}in function}}

\begin{fulllineitems}
\phantomsection\label{\detokenize{js/js-functions:updateSceneValues}}
\pysigstartsignatures
\pysiglinewithargsret
{\sphinxbfcode{\sphinxupquote{\DUrole{n}{updateSceneValues}}}}
{\sphinxparam{\DUrole{n}{array}}\sphinxparamcomma \sphinxparam{\DUrole{n}{scene}}}
{}
\pysigstopsignatures
\sphinxAtStartPar
Funkcja aktualizująca wartości w tablicy dla podanej sceny.
\begin{quote}\begin{description}
\sphinxlineitem{Arguments}\begin{itemize}
\item {} 
\sphinxAtStartPar
\sphinxstyleliteralstrong{\sphinxupquote{array}} (\sphinxstylestrong{dict}) \textendash{} tablica z wartościami

\item {} 
\sphinxAtStartPar
\sphinxstyleliteralstrong{\sphinxupquote{scene}} (\sphinxstylestrong{string}) \textendash{} wskaźnik na aktywną scene

\end{itemize}

\end{description}\end{quote}

\end{fulllineitems}

\index{setSceneValues() (built\sphinxhyphen{}in function)@\spxentry{setSceneValues()}\spxextra{built\sphinxhyphen{}in function}}

\begin{fulllineitems}
\phantomsection\label{\detokenize{js/js-functions:setSceneValues}}
\pysigstartsignatures
\pysiglinewithargsret
{\sphinxbfcode{\sphinxupquote{\DUrole{n}{setSceneValues}}}}
{\sphinxparam{\DUrole{n}{array}}\sphinxparamcomma \sphinxparam{\DUrole{n}{scene}}}
{}
\pysigstopsignatures
\sphinxAtStartPar
Funkcja ustawiająca wartości dla podanej sceny.
\begin{quote}\begin{description}
\sphinxlineitem{Arguments}\begin{itemize}
\item {} 
\sphinxAtStartPar
\sphinxstyleliteralstrong{\sphinxupquote{array}} (\sphinxstylestrong{dict}) \textendash{} tablica z wartościami

\item {} 
\sphinxAtStartPar
\sphinxstyleliteralstrong{\sphinxupquote{scene}} (\sphinxstylestrong{string}) \textendash{} wskaźnik na aktywną scene

\end{itemize}

\end{description}\end{quote}

\end{fulllineitems}

\index{updateSceneShaders() (built\sphinxhyphen{}in function)@\spxentry{updateSceneShaders()}\spxextra{built\sphinxhyphen{}in function}}

\begin{fulllineitems}
\phantomsection\label{\detokenize{js/js-functions:updateSceneShaders}}
\pysigstartsignatures
\pysiglinewithargsret
{\sphinxbfcode{\sphinxupquote{\DUrole{n}{updateSceneShaders}}}}
{\sphinxparam{\DUrole{n}{scene1}}\sphinxparamcomma \sphinxparam{\DUrole{n}{scene2}}}
{}
\pysigstopsignatures
\sphinxAtStartPar
Ukrywa shadery niedostępne w wybranej scenie.
\begin{quote}\begin{description}
\sphinxlineitem{Arguments}\begin{itemize}
\item {} 
\sphinxAtStartPar
\sphinxstyleliteralstrong{\sphinxupquote{scene1}} (\sphinxstylestrong{list}) \textendash{} tablica zawierająca liste shaderów dostępnych na scenie 1

\item {} 
\sphinxAtStartPar
\sphinxstyleliteralstrong{\sphinxupquote{scene2}} (\sphinxstylestrong{list}) \textendash{} tablica zawierająca liste shaderów dostępnych na scenie 2

\end{itemize}

\end{description}\end{quote}

\end{fulllineitems}

\index{hideButton() (built\sphinxhyphen{}in function)@\spxentry{hideButton()}\spxextra{built\sphinxhyphen{}in function}}

\begin{fulllineitems}
\phantomsection\label{\detokenize{js/js-functions:hideButton}}
\pysigstartsignatures
\pysiglinewithargsret
{\sphinxbfcode{\sphinxupquote{\DUrole{n}{hideButton}}}}
{}
{}
\pysigstopsignatures
\sphinxAtStartPar
Ukrywa przycisk do renderowania sceny 1

\end{fulllineitems}

\index{createJSON() (built\sphinxhyphen{}in function)@\spxentry{createJSON()}\spxextra{built\sphinxhyphen{}in function}}

\begin{fulllineitems}
\phantomsection\label{\detokenize{js/js-functions:createJSON}}
\pysigstartsignatures
\pysiglinewithargsret
{\sphinxbfcode{\sphinxupquote{\DUrole{n}{createJSON}}}}
{}
{}
\pysigstopsignatures
\sphinxAtStartPar
Tworzy plik JSON i zapisuje do niego ustawienia aplikacji.

\end{fulllineitems}

\index{loadJSON() (built\sphinxhyphen{}in function)@\spxentry{loadJSON()}\spxextra{built\sphinxhyphen{}in function}}

\begin{fulllineitems}
\phantomsection\label{\detokenize{js/js-functions:loadJSON}}
\pysigstartsignatures
\pysiglinewithargsret
{\sphinxbfcode{\sphinxupquote{\DUrole{n}{loadJSON}}}}
{}
{}
\pysigstopsignatures
\sphinxAtStartPar
Funkcja wczytuje plik JSON i odczytuje zapisane w nim ustawienia aplikacji.
\begin{quote}\begin{description}
\sphinxlineitem{Returns}
\sphinxAtStartPar
sparsowany plik JSON.

\end{description}\end{quote}

\end{fulllineitems}




\renewcommand{\indexname}{Index}
\printindex
\end{document}